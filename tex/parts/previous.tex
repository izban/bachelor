\section{Предыдущие результаты}
\subsection{Наибольшая общая Абелева подстрока}
По постановке задача о поиске наибольшей общей Абелевой подстроке очень похожа на поиск наидлиннейшей общей подстроки~--- очень важную задачу, исследовавшуюся в 70-е года прошлого века. В частности, суффиксное дерево было разработано в процессе работы над несколькими задачами, одной из которых являлся линейный поиск наидлиннейшей общей подстроки.

Задача поиска наибольшей общей Абелевой подстроки (\textit{LCAS}) была сформулирована в 2013 году на конференции StringMasters. Там она была поставлена как открытая задача. 

После этого в 2015 году A.Attabi в \cite{1} предложил субквадратичный алгоритм для случая $\sigma=2$, работающий за $\mathcal{O}(n^2/\log n)$ времени, и решение общего случая за $\mathcal{O}(n^2\sigma)$ времени и $\mathcal{O}(n\sigma)$ памяти.

В 2016 году на конференции SPIRE \cite{4} был улучшен алгоритм 2015 года, уменьшив требование памяти до $\mathcal{O}(n)$, и алгоритм, решающий задачу для случая алфавитов большого размера, работающий за $(\mathcal{O}(n^2 \log^2 n \log^* n), \mathcal{O}(n \log^2 n))$ времени и памяти. 
%TODO расписать больше про каждый алгоритм? про недетерменированный?
%TODO расписать больше мотивации? всякая биоинформатика

Сравнительное времени работы этих алгоритмов можно увидеть в таблице 1.

\begin{table}[H]
\begin{center}
\begin{tabular}{|c|c|c|c|}
\hline
Год & Авторы & Время & Память \\
\hline
2015 & A. Alattabi & $\mathcal{O}(n^2 \sigma)$ & $\mathcal{O}(n \sigma)$ \\
\hline
2016 & S. Grabowski & $\mathcal{O}(n^2 \sigma)$ & $\mathcal{O}(n)$ \\
\hline
2016 & S. Grabowski & $\mathcal{O}(n^2 \log^2 n \log^* n)$ & $\mathcal{O}(n \log^2 n)$ \\
\hline
2017 & Данная работа & $\mathcal{O}(n^2 \log \sigma)$ & $\mathcal{O}(n)$ \\
\hline
\end{tabular}
\end{center}
\caption{Существующие детерменированные алгоритмы поиска НОАП}
\end{table}

\subsection{Наибольший Абелев подквадрат и количество Абелевых подквадратов строки}
Задачи о нахождении наидлиннейшего Абелево подквадрата и их количества были поставлены в 2016 году, с указанием метода, опубликованным в том же году, позволяющего находить большое количество Абелевых характеристик за $\mathcal{O}(n^2 / \log^2 n)$ \cite{5}.

Так же недавно был опубликован новый алгоритм для решения задачи \textit{3-SUM} используя методы аддитивной комбинаторики, решающий частный случай задачи значительно быстрее, чем за квадратичное время~--- за $\mathcal{O}(n^{1.86})$. Так же они показали, как отвечать на запросы \textit{histogram queries} за $\mathcal{O}(1)$ после предподсчета за $\mathcal{O}(n^{1.86})$.