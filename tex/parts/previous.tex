\section{Предыдущие результаты}
\subsection{Наибольшая общая Абелева подстрока}
Задача поиска наибольшей общей Абелевой подстроки (\textit{LCAS}) была сформулирована в 2013 году на конференции StringMasters. Там она была поставлена как открытая задача.

После этого в 2015 году чуваки предложили субквадратичный алгоритм для случая $\sigma=2$, работающий за $\mathcal{O}(n^2/\log n)$ времени, и решение общего случая за $\mathcal{O}(n^2\sigma)$ времени и $\mathcal{O}(n\sigma)$ памяти.

В 2016 году на конференции SPIRE \cite{1} был улучшен алгоритм 2015 года, уменьшив требование памяти до $\mathcal{O}(n)$, и алгоритм, решающий задачу для случая алфавитов большого размера, работающий за $(\mathcal{O}(n^2 \log^2 n \log^* n), \mathcal{O}(n \log^2 n))$ времени и памяти.

\subsection{Наибольший Абелев подквадрат и количество Абелевых подквадратов строки}
Абелев подквадрат~--- подстрока, являющаяся Абелевым квадратом.

Задачи о нахождении наидлиннейшего Абелево подквадрата и их количества были поставлены в 2016 году, с указанием метода, опубликованным в том же году, позволяющего находить большое количество Абелевых характеристик за $\mathcal{O}(n^2 / \log^2 n)$.

Так же в 2016 году был опубликован новый алгоритм для решения задачи \textit{3-SUM} используя методы аддитивной комбинаторики, решающий частный случай задачи значительно быстрее, чем за квадратичное время~--- за $\mathcal{O}(n^{1.86})$. Так же они показали, как отвечать на запросы \textit{histogram queries} за $\mathcal{O}(1)$ после предподсчета за $\mathcal{O}(n^{1.86})$.