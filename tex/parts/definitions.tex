\section{Используемые определения}

\begin{definition}
$P(s)$~--- вектор Парея, вектор частот символов строки $s$.
\end{definition}

\begin{definition}
$a \equiv b$, если $P(a) = P(b)$~--- Абелева эквивалентность двух строк. Две строки Абелево эквивалентны, если существует перестановка, переводящая одну из строк в другую.
\end{definition}

\begin{definition}
$s$~--- Абелев квадрат, если $s=ab$, где $a \equiv b$.
\end{definition}

\begin{definition}
Абелев подквадрат строки $s$~--- подстрока строки $s$, являющаяся Абелевым квадратом.
\end{definition}

\begin{definition}
\texttt{w.h.p.}~--- \textit{with high probability}, решение, с большой вероятностью работающее за такое время.
\end{definition}

\begin{definition}
Будем говорить, что алгоритм работает за $\langle \mathcal{O}(f(n)), \mathcal{O}(g(n)) \rangle$, если он работает, используя $\mathcal{O}(f(n))$ времени и  $\mathcal{O}(g(n))$ памяти.
\end{definition}

\begin{definition}
\textit{НОАП} (\textit{LCAF})~--- наибольшая общая Абелева подстрока (longest common Abelian factor).
\end{definition}

\begin{problem}
$3SUM^+$: дано три множества $A, B, C$, нужно найти три числа $a \in A, b \in B, c \in C$ такие, что $a+b=c$.
\end{problem}

\begin{problem}
Поиск НОАП: даны две строки $a, b \in \Sigma^n$. Нужно найти наидлиннейшую строку $x$ такую, что $xs_a \equiv a, xs_b \equiv b$ для некоторых $s_a, s_b$.
\end{problem}

\begin{problem}
Число Абелевых подквадратов: дана строка $s=s_0s_1 \ldots s_{n-1}$ из символов двух типов, $a$ и $b$. Нужно найти количество различных ее подстрок $s_{i \cdots j}$, являющихся Абелевыми квадратами.
\end{problem}

\begin{problem}
\textit{Jumbled indexing}~--- задача проверки, содержит ли данный текст $T$ подстроку, Абелево эквивалентную шаблону $S$.
\end{problem}



%\begin{problem}

%\end{problem}

%Еще что-нибудь наверн