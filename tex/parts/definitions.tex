\section{Используемые определения}
$P(s)$~--- вектор Парея, вектор частот символов строки $s$.

$a \equiv b$, если $P(a) = P(b)$~--- Абелева эквивалентность двух строк. Две строки Абелево эквивалентны, если существует перестановка, переводящая одну из строк в другую.

$s$~--- Абелев квадрат, если $s=ab$, где $a \equiv b$.

\textit{Jumbled indexing}~--- задача проверки, содержит ли данный текст $T$ подстроку, Абелево эквивалентную шаблону $S$.

\textit{3-SUM}~--- задача проверки, существуют ли $x \in X, y \in Y, z \in Z$ такие, что $x+y=z$.

Еще что-нибудь наверн