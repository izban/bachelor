\section{Используемые определения}

Введем набор определений, которые будут использоваться по ходу работы. 

\begin{definition}
Алфавит $\Sigma=\{c_1, c_2, \ldots, c_\sigma\}$~--- конечное множество символов. Число символов алфавита $|\Sigma|=\sigma$.
\end{definition}

\begin{definition}
Обозначим за $|s|_{c_1}$ количество символов $c_1$ в строке $s$.
\end{definition}


\begin{definition}
Вектор Парея $\mathcal{P}(s)=(|s|_{c_1}, |s|_{c_2}, \ldots, |s|_{c_\sigma})$~--- вектор частот символов строки $s$.
\end{definition}

\begin{definition}
Две строки $a$ и $b$ Абелево эквивалентны, если существует перестановка $\pi$, переводящая одну из строк в другую, $\pi(a)=b$.
\end{definition}

\begin{definition}
Строка $s$ является Абелевым квадратом, если существуют две Абелево эквивалентные строки $a \equiv b$, что $s = ab$.
%$s$~--- Абелев квадрат, если $s=ab$, где $a \equiv b$.
\end{definition}

\begin{definition}
Абелевым подквадратом строки $s$ назовем подстроку строки $s$, являющуюся Абелевым квадратом.
\end{definition}

\begin{definition}
Событие является \texttt{w.h.p.}~--- \textit{with high probability}, если вероятность такого исхода зависит от какого-то параметра $n$, и эта вероятность стремится к единицы при стремлении $n$ к бесконечности.
\end{definition}
%TODO WARNING переписать?

\begin{definition}
Будем говорить, что алгоритм работает за $\langle \mathcal{O}(f(n)), \mathcal{O}(g(n)) \rangle$, если он работает, используя $\mathcal{O}(f(n))$ времени и  $\mathcal{O}(g(n))$ памяти.
\end{definition}

\begin{definition}
\textit{НОАП} (\textit{LCAF})~--- наибольшая общая Абелева подстрока (longest common Abelian factor).
\end{definition}

Далее сформулируем некоторые задачи, которые используются в работе.

\begin{problem}
$3SUM^+$. Дано три множества целых чисел $A, B, C$. Нужно найти три числа $a \in A, b \in B, c \in C$ такие, что $a+b=c$.
\end{problem}

\begin{problem}
%Поиск НОАП: даны две строки $a, b \in \Sigma^n$. Нужно найти наидлиннейшую строку $x$ такую, что $xs_a \equiv a, xs_b \equiv b$ для некоторых $s_a, s_b$.
Поиск НОАП. Даны две строки $a, b$ над алфавитом $\Sigma$. Нужно найти подстроку $x$ строки $a$ и подстроку $y$ строки $b$, что $x \equiv y$, а длина подстрок максимальна.
\end{problem}

\begin{problem}
Подсчет числа Абелевых подквадратов. Дана строка $s=s_0s_1 \ldots s_{n-1}$ над алфавитом мощности 2, $\Sigma = \{c_0, c_1\}$. Нужно найти количество различных ее подстрок $s_{i \cdots j}$, являющихся Абелевыми квадратами.
\end{problem}

\begin{problem}
\textit{Jumbled indexing}. Задача проверки того, содержит ли данный текст $T$ подстроку, Абелево эквивалентную заданному шаблону $S$.
\end{problem}



%\begin{problem}

%\end{problem}

%Еще что-нибудь наверн