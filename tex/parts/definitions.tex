\section{Используемые определения}

\begin{definition}
$P(s)$~--- вектор Парея, вектор частот символов строки $s$.
\end{definition}

\begin{definition}
$a \equiv b$, если $P(a) = P(b)$~--- Абелева эквивалентность двух строк. Две строки Абелево эквивалентны, если существует перестановка, переводящая одну из строк в другую.
\end{definition}

\begin{definition}
$s$~--- Абелев квадрат, если $s=ab$, где $a \equiv b$.
\end{definition}

\begin{definition}
Абелев подквадрат строки $s$~--- подстрока строки $s$, являющаяся Абелевым квадратом.
\end{definition}

%\begin{problem}
%\textit{Jumbled indexing}~--- задача проверки, содержит ли данный текст $T$ подстроку, %Абелево эквивалентную шаблону $S$.
%\end{problem}

%\begin{problem}
%$3SUM^+$: дано три множества $A, B, C$, нужно найти три числа $a \in A, b \in B, c \in %C$ такие, что $a+b=c$.
%\end{problem}

%\begin{problem}

%\end{problem}

Еще что-нибудь наверн