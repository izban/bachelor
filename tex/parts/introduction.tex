%\newpage
%\section{Введение}

В последнее время тема Абелевой эквивалентности строк стала активно обсуждаться в научном сообществе.
Первые определения, такие как вектор числа встреч каждого символа в подстроке, называемый вектором Парея,
и первые сформулированные задачи на эту тему, такие как нахождение подстроки
с заданным вектором Парея, называемой \textit{jumbled indexing}, были предложены ещё в 60-х годах прошлого века.
Но только около двадцати лет назад стали появляться первые результаты в этой области.
С тех пор результатов с каждым годом становилось всё больше, и, в конце концов, Абелева эквивалентность выделилась в
самостоятельную подобласть.

Задачи, связанные с Абелевой эквивалентностью, встречаются в большом числе различных областей, связанных с информатикой.
Например, представленная выше задача \textit{jumbled indexing} находит применение в бионформатике
при решении задач \textit{mass spectrometry} и \textit{gene clusters}.
Кроме того, Абелево совпадение слов, иначе, совпадение векторов Парея,
может применяться как критерий эвристического фильтра для поиска шаблона в тексте,
а также поиска шаблона с ошибками.

В главе 1 мы вводим основные определения, используемые в работе, ставим решаемые задачи и представляем
известные на сегодняшний день результаты.

В главе 2 мы предлагаем новые алгоритмы для решения задачи о поиске числа Абелевых подквадратов
и задачи о нахождении наибольшей общей Абелевой подстроки.
В этой же главе мы приводим теоретические оценки к поставленным задачам.

В главе 3 мы приводим экспериментальные результаты алгоритмов, предложенных в главе 2.

%TODO мало, разбавить бы