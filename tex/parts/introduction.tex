%\newpage
%\section{Введение}

В последнее время стало появляться множество работ на тему Абелевой эквивалентности строк. Первые статьи на эту тему (?я нашел?) опубликованы около пятнадцати лет, и с тех пор наука шагнула далеко вперед в этом направлении.

Задачи данного типа встречаются в широком классе областей. Так, \textit{jumbled indexing} находит применение в бионформатике, при решении задач \textit{mass spectrometry} и \textit{gene clusters}. Кроме того, Абелево совпадение слов~--- хороший критерий эвристического фильтра для \textit{exact matching} и поиска с ошибками.

В главе 1 будут рассмотрены основные определения, используемые в работе, и известные на сегодняшний день результаты.

В главе 2 будут предложены новые алгоритмы и оценки на задачи по данной теме.

В главе 3 будут приведены практические результаты предложенных алгоритмов.
% Это одна из самых сложных частей. Здесь надо будет написать про мотивацию, описать предыдушие работы и поставить задачу.