%\section{Заключение}
% Здесь говоришь какой ты молодец и сколько всего сделал
В рамках данной работы была сведена задача о числе Абелевых подквадратов в бинарной строке к монотонному и линейно ограниченному случаю задачи $3SUM^+$. У этой задачи недавно было опубликовано неожиданное асимптотически очень хорошее решение. Этот алгоритм был реализован, и мы пришли к выводу, что его реализация не очень эффективна, проигрывая в несколько раз простому алгоритму с более плохой асимптотикой, но хорошей константой. Несмотря на это, заметен потенциал предложенного метода, и он вероятно может быть улучшен до производительности, действительно применимой на практике.

Кроме того, в этой работе было проведено исследование задачи НОАП. Был разработан новый алгоритм, существенно улучшающий известные теоретические результаты для задачи в общей формулировке без ограничений на размер алфавита. По сравнению с предыдущим лучшим результатом для больших алфавитов, алгоритм не только асимптотически выигрывает по затрачиваемому времени и памяти, но и более прост для понимания. Несмотря на хороший результат, остается открытым вопрос поиска более быстрых детерменированных алгоритмов. Так же показано на практике, что предложенный алгоритм значительно превосходит известные ранее алгоритмы.

Был проведен анализ поведения НОАП для случайных бинарных строк~--- вопрос, исследовавшийся ранее. Экспериментально показана линейная зависимость матожидания НОАП от длины строк, и теоретически обоснованы линейные оценки сверху и снизу. Поиск точных оценок остался в качестве нерешенной задачи.

