%\section{Заключение}
% Здесь говоришь какой ты молодец и сколько всего сделал
В рамках данной работы была сведена задача о числе Абелевых подквадратов к монотонному и линейно ограниченному случаю задачи $3SUM^+$. У этой задачи недавно было опубликовано неожиданное асимптотически очень хорошее решение. Это решение было реализовано, и мы пришли к выводу, что его реализация не очень эффективна, проигрывая в несколько раз простым алгоритмам с более плохой асимптотикой, но хорошей константой. Несмотря на это, заметен потенциал предложенного метода, и он вероятно может быть улучшен до производительности, действительно применимой на практике.

Кроме того, в этой работе я провел исследование задачи о НОАП. Был разработан новый алгоритм, существенно улучшающий известные теоретические результаты для задачи в общей формулировке без ограничений на размер алфавита. По сравнению с предыдущим лучшим результатом для больших алфавитов, алгоритм не только асимптотически выигрывает по затрачиваемому времени и памяти, но и более прост для понимания. Несмотря на хороший результат, остается открытым вопрос поиска более быстрых детерменированных алгоритмов.

Был проведен анализ поведения НОАП для случайных бинарных строк~--- вопрос, исследовавшийся ранее. Экспериментально показана линейная зависимость матожидания НОАП от длины строк, и теоретически обоснованы линейные оценки сверху и снизу. Поиск точных оценок поставлен нерешенной задачей.

