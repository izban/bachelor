\documentclass[hyperref=unicode,graphics=pdflatex,12pt]{beamer}

\usepackage[T2A]{fontenc}
\usepackage[utf8]{inputenc}
\usepackage[russian]{babel}


\mode<presentation>
{
  \setbeamercovered{invisible}
}

\usepackage{amssymb,amsthm,amsmath,amsfonts,array,pstcol,graphicx,pst-node,comment,rotating,ccaption,beamerthemesplit}

\def\sgn{\text{sgn}\,}

\useoutertheme{infolines}
\setbeamertemplate{theorems}[numbered]
\setbeamertemplate{footline}[page number]{}
\setbeamertemplate{headline}{}

\deftranslation{Theorem}{Теорема}
\deftranslation{Lemma}{Лемма}
\deftranslation{Definition}{Определение}

\begin{document}

\title{Поиск абелевых строк наибольшей длины}
\author{И.~Збань\\
Научный руководитель: В.~Аксёнов}

\date{\today}
\institute{\includegraphics[width=0.3\textwidth]{pics/itmo.png}}
\frame{\titlepage}                                   

\begin{comment}
\begin{frame}{}
  \tableofcontents
\end{frame}
\end{comment}

\section{Введение}

\begin{frame}{Постановка задачи}
\hspace{0.5cm}
Задача: Нахождение наибольшей общей абелевой подстроки и поиск абелевых подквадратов.

\vspace{0.5cm}
\hspace{0.5cm}
\onslide<2->
Две строки Абелево эквививалентны, если есть перестановка, переводящая одну строку в другую.

\vspace{0.5cm}
\hspace{0.5cm}
\onslide<3->
Абелев подквадрат~--- подстрока, представимая как конкатенация двух абелево эквивалентных строк.


\end{frame}


\begin{frame}{Почему эта тема}
\hspace{0.5cm}
\onslide<1->
\alert{Мотивация:}
\begin{itemize}
\item<2-> Быстроразвивающаяся область, много публикаций за последнее время
\item<3-> Актуальность: подзадачи в бионформатике (gene clusters), фильтры в задаче поиска образца
\item<4-> Близость с известной задачей 3SUM
\end{itemize}
\end{frame}

\begin{frame}{Краткое описание}
\hspace{0.5cm}
Работа состоит из следующих пунктов:
\begin{itemize}
\item<2-> Оценка алгоритма решения 3SUM+ для монотонных множеств на примере задачи о количестве абелевых подквадратов
\item<3-> Анализ задачи LCAF для частного случая бинарного алфавита
\item<4-> Решение задачи LCAF для общего случая
\end{itemize}
\end{frame}

\begin{frame}{Количество абелевых подквадратов}
Задача о количестве абелевых подквадратов сводится к $3SUM^+$
\vspace{0.5cm}

\onslide<2-> A=B=\{($c_a(i)$, $c_b(i)$)\}, C=\{$2c_a(i)$,$2c_b(i)$\}
\vspace{0.5cm}

где $c_a(i), c_b(i)$~--- количество букв $a$ и $b$ на префиксе длины $i$
\vspace{0.5cm}

\onslide<3-> и число подстрок~--- (\#3$SUM^+$(A,B,C)-(n+1))/2

\end{frame}

\begin{frame}{Сравнение алгоритмов на простой строке}
Картиночка на которой видно, что квадрат работает быстро, а 1.86~--- медленно
\end{frame}

\begin{frame}{Сравнение алгоритмов на случайном тесте}
Картиночка, на которой видно, что квадрат работает быстро, а 1.86~--- оооочень медленно
\end{frame}

\begin{frame}{binary LCAF}
\includegraphics[scale=0.6]{pics/avlcas.png}
\end{frame}

\begin{frame}{binary LCAF}
\vspace{0.5cm}
В работе доказана оценка сверху, что LCAF ограничена линейной функцией, тем самым опровергнута посылка из первоисточника

\vspace{0.5cm}
Предложено сведение задачи к известному алгоритму, получая оптимальное на данный момент решение за $n^{1.86}$.
\end{frame}

\begin{frame}{general LCAF}
Используя персистентные деревья с limited node copying предложен алгоритм LCAF в общем случае за $(\mathcal{O}(n^2 \log \sigma), \mathcal{O}(n))$

\vspace{0.5cm}
Идея~--- построить персистентный массив вектора Парея для строки-конкатенации обеих данных строк, посчитать некий хеш от каждого корня, и проверить, были ли одинаковые версии, соответствующие обеим строкам
\end{frame}


\begin{frame}{Схема вычисления хеша}
Тут какая-то непонятная картинка
\end{frame}

\begin{frame}{Сравнение алгоритмов LCAF}
\begin{tabular}{|c|c|c|c|}
\hline
Год & Авторы & Время & Память \\
\hline
2013 & StringMasters & - & - \\
\hline
2015 & Кто-то & $\mathcal{O}(n^2 \sigma)$ & $\mathcal{O}(n \sigma)$ \\
\hline
2016 & Кто-то & $\mathcal{O}(n^2 \sigma)$ & $\mathcal{O}(n)$ \\
\hline
2016 & SPIRE & $\mathcal{O}(n^2 \log^2 n \log^* n)$ & $\mathcal{O}(n \log^2 n)$ \\
\hline
2017 & Я & $\mathcal{O}(n^2 \log \sigma)$ & $\mathcal{O}(n)$ \\
\hline
\end{tabular}
\end{frame}

\begin{frame}{Вопросы?}
\begin{center}
Спасибо за внимание.
\end{center}
\end{frame}
\end{document}